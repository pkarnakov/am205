\documentclass[12pt]{article}
\usepackage{fullpage,amsmath,amsfonts,mathpazo,microtype,nicefrac}
\usepackage{xspace}

\usepackage[
  pdfusetitle=true, colorlinks=true, urlcolor=color2, linkcolor=color2,
  bookmarksdepth=1,
]{hyperref}

\usepackage{cleveref}
\crefname{equation}{}{}
\usepackage{titlesec}
% Section (problem) heading.
\titleformat{\section}{\large\bfseries}{}{0em}{}
\renewcommand{\thesection}{P\arabic{section}}
\crefname{section}{}{}
% Subsection (part) heading.
\titleformat{\subsection}[runin]{\normalfont\bfseries}{}{0em}{}
\renewcommand{\thesubsection}{(\alph{subsection})}
\crefname{subsection}{}{}

% Custom title page.
\makeatletter
\renewcommand\maketitle{
{\raggedright{\Large\bfseries\@title}\\[2ex]}}
\makeatother

% Macro definitions.
\newcommand{\N}{\mathbb{N}}
\newcommand{\Z}{\mathbb{Z}}
\newcommand{\Q}{\mathbb{Q}}
\newcommand{\R}{\mathbb{R}}
\newcommand{\p}{\partial}
\renewcommand{\vec}[1]{\mathbf{#1}}
\newcommand{\vx}{\vec{x}}
\newcommand{\vp}{\vec{p}}
\newcommand{\Trans}{\mathsf{T}}
\newcommand{\Optional}{Optional.}
\newcommand{\Part}[1]{\textbf{(#1)}}
\newcommand{\GitRoot}{https://github.com/pkarnakov/am205/tree/main}
\newcommand{\HrefPublic}[2]{\href{\GitRoot/\detokenize{#1}}{\normalfont\texttt{[\detokenize{#2}]}}}
\newcommand{\Pts}[1]{~[\textbf{#1 pts}]}

% Colors.
\usepackage{xcolor}
\definecolor{color0}{HTML}{FF1F5B}
\definecolor{color1}{HTML}{00CD6C}
\definecolor{color2}{HTML}{009ADE}
\definecolor{color3}{HTML}{AF58BA}
\definecolor{color4}{HTML}{FFC61E}
\definecolor{color5}{HTML}{F28522}
\definecolor{color6}{HTML}{A0B1BA}
\definecolor{color7}{HTML}{A6761D}
\definecolor{color8}{HTML}{E9002D}
\definecolor{color9}{HTML}{FFAA00}
\definecolor{color10}{HTML}{00B000}

\usepackage{graphicx}
\usepackage[section]{placeins}


\title{AM205 HW0. Introduction}

\begin{document}

\maketitle

\section{P1. Chebyshev polynomials}

The Chebyshev polynomials $T_k(x)$ can be defined using the recursive relation
\[
T_k(x) = 2xT_{k-1}(x) - T_{k-2} (x)
\]
and $T_0(x)=1$, $T_1(x)=x$. Evaluate and plot the Chebyshev polynomial of degree
5 at 101 evenly spaced points in the interval $x\in [-1,1]$. Draw a 3D surface
plot of the function $T_3(x)T_5(y)$ on a $101\times 101$ grid in the domain
$(x,y) \in [-1,1]^2$.

\section{P2. Square root}
Use the iteration
\[
x_{k+1} = \frac{1}{2} \left( x_k + \frac{a}{x_k} \right)
\]
to approximate $\sqrt{a}$. This is known as Heron's
formula and it is equivalent to
the Newton--Raphson method for the function $f(x)=x^2-a$. Choose an initial
starting value of $x_0=a$ and iterate until $|x_{k+1} - x_k|<\epsilon$ for
some tolerance $\epsilon$. Determine the number of iterations required to
compute $\sqrt{5}$ for the cases of $\epsilon=10^{-3}$ and
$\epsilon=10^{-9}$.

\section{P3. Finite differences}

\subsection{(a)}
Let $f(x)=\tan x$ and consider the finite-difference approximation
\[
  f'_a(x;h) = \frac{f(x+h)-f(x-h)}{2h}.
\]
Make a log--log plot of the relative error
$E=|f'(x) - f'_a(x;h)| / |f'(x)|$ at $x=1$
as a function of $h$ for $h=10^{-k}$, using
$k=1,1.5,\ldots, 16$. Use linear regression to fit the
relative error to the straight line
\[
  \log E = \log C + q \log h
\]
for some constants $C$ and $q$. Show that $q\approx 2$,
meaning that the approximation is second-order accurate.

\subsection{(b)}
Repeat the analysis for the approximation
\[
f'_b(x;h) = \frac{-11f(x) + 18 f(x+h) - 9f(x+2h) + 2f(x+3h)}{6h}
\]
and determine the rate of convergence $q$.

\section{P4. Calculation of Pi}
In the first lecture we discussed Archimedes' method of calculating~$\pi$
from perimeters of inscribed and superscribed regular
polygons.
Areas of the polygons can serve the same purpose.
Consider a circle of unit radius.
Let $a_n$ and $b_n$ be the areas of inscribed and superscribed regular polygons
with $3 \times 2^n$ sides, respectively.

\subsection{(a)}
The case of $n=0$ therefore corresponds to inscribed and superscribed
equilateral triangles. Use geometry to show that $a_0=\frac{3}{4}\sqrt{3}$ and
$b_0=3\sqrt{3}$.

\subsection{(b)}
Show that
\[
\frac{2}{b_{n+1}} = \frac{1}{a_{n+1}} + \frac{1}{b_n}, \qquad a_{n+1}^2 = a_nb_n
\]
and write a program to evaluate $(a_n,b_n)$ for $n=0,1,\ldots,40$.
In addition, calculate $c_n=\frac{1}{2}(a_n+b_n)$.

\subsection{(c)}
Make a log-linear plot of the absolute errors $|a_n-\pi|$ and
$|c_n-\pi|$ as a function of $n$. How fast do these two sequences
converge to $\pi$? Is there a difference in the convergence rate
between $a_n$ and $c_n$?

\end{document}
