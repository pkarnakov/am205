\documentclass[12pt]{article}
\usepackage{fullpage,amsmath,amsfonts,mathpazo,microtype,nicefrac}
\usepackage{xspace}

\usepackage[
  pdfusetitle=true, colorlinks=true, urlcolor=color2, linkcolor=color2,
  bookmarksdepth=1,
]{hyperref}

\usepackage{cleveref}
\crefname{equation}{}{}
\usepackage{titlesec}
% Section (problem) heading.
\titleformat{\section}{\large\bfseries}{}{0em}{}
\renewcommand{\thesection}{P\arabic{section}}
\crefname{section}{}{}
% Subsection (part) heading.
\titleformat{\subsection}[runin]{\normalfont\bfseries}{}{0em}{}
\renewcommand{\thesubsection}{(\alph{subsection})}
\crefname{subsection}{}{}

% Custom title page.
\makeatletter
\renewcommand\maketitle{
{\raggedright{\Large\bfseries\@title}\\[2ex]}}
\makeatother

% Macro definitions.
\newcommand{\N}{\mathbb{N}}
\newcommand{\Z}{\mathbb{Z}}
\newcommand{\Q}{\mathbb{Q}}
\newcommand{\R}{\mathbb{R}}
\newcommand{\p}{\partial}
\renewcommand{\vec}[1]{\mathbf{#1}}
\newcommand{\vx}{\vec{x}}
\newcommand{\vp}{\vec{p}}
\newcommand{\Trans}{\mathsf{T}}
\newcommand{\Optional}{Optional.}
\newcommand{\Part}[1]{\textbf{(#1)}}
\newcommand{\GitRoot}{https://github.com/pkarnakov/am205/tree/main}
\newcommand{\HrefPublic}[2]{\href{\GitRoot/\detokenize{#1}}{\normalfont\texttt{[\detokenize{#2}]}}}
\newcommand{\Pts}[1]{~[\textbf{#1 pts}]}

% Colors.
\usepackage{xcolor}
\definecolor{color0}{HTML}{FF1F5B}
\definecolor{color1}{HTML}{00CD6C}
\definecolor{color2}{HTML}{009ADE}
\definecolor{color3}{HTML}{AF58BA}
\definecolor{color4}{HTML}{FFC61E}
\definecolor{color5}{HTML}{F28522}
\definecolor{color6}{HTML}{A0B1BA}
\definecolor{color7}{HTML}{A6761D}
\definecolor{color8}{HTML}{E9002D}
\definecolor{color9}{HTML}{FFAA00}
\definecolor{color10}{HTML}{00B000}

\usepackage{graphicx}
\usepackage[section]{placeins}


\title{AM205 HW1. Data fitting}

\newcommand{\ZipName}{\texttt{\detokenize{am205_hw1_data.zip}}}

\begin{document}

\maketitle

\section{P1. Polynomial approximation of the gamma function\Pts{10}}

The \href{https://en.wikipedia.org/wiki/Gamma_function}{gamma function}
is defined as
\begin{equation*}
  \Gamma(x) = \int_0^\infty t^{x-1} e^{-t}\,dt
\end{equation*}
and satisfies $(n-1)! = \Gamma(n)$ for integers $n$.\\
\textbf{Note:} In Python, the gamma function is available in the
\href{https://docs.scipy.org/doc/scipy/reference/generated/scipy.special.gamma.html}{\texttt{scipy.special}}
module.

\subsection{(a)\Pts{3}}
Construct an approximation $g(x)$ to the gamma function as the Lagrange
polynomial of the following points:
\begin{center}
  \begin{tabular}{c|r|r|r|r|r}
    $n$ & 1 & 2 & 3 & 4 & 6 \\
    \hline
    $\Gamma(n)$ & 1 & 1 & 2 & 6 & 120
  \end{tabular}
\end{center}
Write the polynomial as $g(x)=\sum_{k=0}^4 g_k x^k$.
Report the values of the coefficients $g_k$.

\subsection{(b)\Pts{3}}
Construct another approximation~$h(x)$ to the gamma function by first
calculating the fourth order polynomial~$p(x)$ that interpolates the points
$(n,\log(\Gamma(n)))$ for $n=1,2,3,4,6$. Then define the approximation as
$h(x)=\exp(p(x))$.
Report the values of the coefficients of $p(x)$.

\subsection{(c)\Pts{2}}
Plot values of $\Gamma(x)$, $g(x)$, and $h(x)$
and the relative errors $|\Gamma(x)-g(x)|/\Gamma(x)$ and
$|\Gamma(x)-h(x)|/\Gamma(x)$ on the interval $1\le x \le 6$.
Use a logarithmic scale for the error.

\subsection{(d)\Pts{2}}
Calculate the maximum relative error between $\Gamma(x)$ and $g(x)$ on the
interval $1\le x\le 6$ by evaluating the functions on a uniform grid with 1001
points. Repeat this for $\Gamma(x)$ and $h(x)$. Which of the two approximations
is more accurate?

\section{P2. Error bounds with Lagrange polynomials\Pts{15}}

\subsection{(a)\Pts{3}}
Let $f(x)=e^{-4x}+e^{3x}$.
Write a program to calculate and plot the
Lagrange polynomial $p_{n-1}(x)$ of $f(x)$ at the Chebyshev points $x_j
= \cos ((2j-1)\pi/2n)$ for $j = 1,\ldots,n$. For $n=4$, over the range
$[-1,1]$, plot $f(x)$ and Lagrange polynomial $p_3(x)$.

\subsection{(b)\Pts{4}}
Recall from the lectures that the infinity norm for a function $g$ on $[-1,1]$
is defined as $\|g\|_\infty = \max_{x\in[-1,1]} |g(x)|$. Calculate
$\|f-p_3\|_\infty$ by sampling the function at 1001 equally-spaced points over
$[-1,1]$.

\subsection{(c)\Pts{6}}
Recall the interpolation error formula from the lectures,
\begin{equation*}
  f(x)-p_{n-1}(x) = \frac{f^{(n)}(\theta)}{n!} (x-x_1)(x-x_2)\ldots(x-x_n)
\end{equation*}
for some $\theta \in [-1,1]$. Use this formula to derive an upper bound for
$\|f-p_{n-1}\|_\infty$ for any positive integer $n$. Your bound should be a
mathematical formula, and should not rely on numerical sampling.

\subsection{(d)\Pts{2}}
Find a cubic polynomial $p_3^\dagger$ such that
$\|f-p_3^\dagger\|_\infty<\|f-p_3\|_\infty$.

\section{P3. Condition number of a matrix\Pts{8}}
For a $2\times 2$ invertible matrix $A$, define the condition number to be
$\kappa(A) = \|A\|\,\|A^{-1}\|$
as discussed in the lectures. Assume that the matrix norm is defined using the
Euclidean vector norm.

\subsection{(a)\Pts{2}}
Find two $2\times 2$ invertible matrices $B$ and $C$ such that 
$\kappa(B+C) < \kappa(B) + \kappa(C)$.

\subsection{(b)\Pts{2}}
Find two $2\times 2$ invertible matrices $B$ and $C$ such that
$\kappa(B+C) > \kappa(B) + \kappa(C)$.

\subsection{(c)\Pts{4}}
Suppose that $Q$ is an orthogonal matrix, i.e. $Q^\Trans=Q^{-1}$,
and $\alpha\in\mathbb{R}$.
Find $\kappa(QA)$ in terms of $\kappa(A)$ and $Q$.
Find $\kappa(\alpha A)$ in terms of $\kappa(A)$ and $\alpha$.

\section{P4. Periodic cubic splines\Pts{15}}

In the lectures we discussed the
construction of cubic splines to interpolate between a number of control
points. We found that it was necessary to impose additional constraints at
the end points of the spline in order to have enough constraints to
determine the cubic spline uniquely. Here, we examine the construction of
cubic splines on a periodic interval $t\in [0,4)$, where $t=0$ is
equivalent to $t=4$.
Working in a periodic interval simplifies the spline construction
and requires no additional constraints. \\
\textbf{Note:}
Do not use existing modules, such as \verb`scipy.interpolate`, for constructing
the spline.
However, you may use other modules for solving linear systems
and numerical integration (e.g. \verb`numpy.linalg.solve` and
\verb`scipy.integrate`).

\subsection{(a)\Pts{5}}
Consider four points $(t,x)=(0,0), (1,4), (2,0), (3,-4)$.  Construct a cubic
spline $s_x(t)$ that is piecewise cubic in the four intervals $[0,1)$, $[1,2)$,
$[2,3)$, and $[3,4)$. At $t=0,1,2,3$ the cubics should match the control points,
giving eight constraints. At $t=0,1,2,3$ the first and second derivatives should
match, giving and additional eight constraints and allowing $s_x(t)$ to be
uniquely determined.

\subsection{(b)\Pts{1}}
Plot $s_x(t)$ and $4\sin(t\pi/2)$ on the interval $[0,4)$ and show that they are
similar.

\subsection{(c)\Pts{6}}
Construct a second cubic spline $s_y(t)$ that goes through the four points
$(t,y)=(0,2), (1,0), (2,-2), (3,0)$. Plot $s_y(t)$ and $2\cos(t\pi/2)$ on the
interval $0 \le t <4$ to see how similar they are.

\subsection{(d)\Pts{3}}
In the $xy$-plane, plot the parametric curve $(s_x(t),s_y(t))$ for $t\in[0,4)$.
Calculate the area enclosed by the parametric curve to at least five decimal
places. Use the area to estimate $\pi$ from the relationship $A=r_1r_2\pi$ where
$r_1=2$ and $r_2=4$.

\section{P5. Image reconstruction from low light\Pts{24}}

\newcommand{\tpA}{{\texttt{0258}}\xspace}
\newcommand{\tpB}{{\texttt{0646}}\xspace}
\newcommand{\tpC}{{\texttt{0704}}\xspace}
\newcommand{\tpD}{{\texttt{0927}}\xspace}

In the archive \ZipName, you will find a directory called \verb`p5_fragments`
that contains several photo fragments of a scene with different objects
taken from
\href{http://www.youtube.com/watch?v=Er1d5QV3LHU}{a bird feeder station in the Netherlands}.
The filename of each image (e.g. \verb`0258_frag0.png`) consists of two parts.
The first part is a timestamp (hours and minutes).
The second part denotes one of the four fragments showing different objects.
Images \tpA, \tpB, and \tpC were taken in low light,
while image \tpD was taken in regular light.

\begin{figure}[h!]
  \centering
  \includegraphics[height=2.3in]{p5_scene.jpg}
  \includegraphics[height=2.3in]{p5_all_frags.jpg}
  \caption{Full scene at time \texttt{0927} (left) and all extracted fragments
  (right).}
\end{figure}

The size of each image is $256\times256$ pixels.
Each pixel in the image can be represented as
a vector $\vp\in\mathbb{R}^3$ containing three intensity values between 0
and 1 for the red, green, and blue components.
Assume that all pixels in fragments having the same timestamp are indexed
and $K_\mathrm{all}$ is the set of all possible indices,
so that $|K_\mathrm{all}|=4\cdot 256^2$.
Let $K_0$, $K_1$, $K_2$, $K_3\subset K_\mathrm{all}$ denote the subsets
of indices corresponding to fragments 0, 1, 2, and 3 respectively,
so that $|K_0|=|K_1|=|K_2|=|K_3|=256^2$.
Then, let $\vp^A_k$, $\vp^B_k$, $\vp^C_k$, and $\vp^D_k$
be the $k$-th pixel of images with timestamps 
\tpA, \tpB, \tpC, and \tpD respectively.

\subsection{(a)\Pts{13}}
Consider reconstructing the regular-light photo~$\vp^D$
from the three low-light photos~$\vp^A$, $\vp^B$, and $\vp^C$
using the following model
\begin{equation}
 \label{e_color_model}
 \vp_k = F^A \vp^A_k + F^B \vp^B_k + F^C \vp^C_k + \vp_\text{const},
\end{equation}
where $F^A$, $F^B$, and $F^C$ are $3\times 3$ matrices and
$\vp_\text{const}\in\mathbb{R}^3$.
Find a least-squares fit for the unknown
$F^A$, $F^B$, $F^C$, and $\vp_\text{const}$ using pixels from fragments~0 and~1.
Specifically, you should minimize the error
\begin{equation*}
  S_{ABC}(K) = \frac{1}{|K|} \sum_{k\in K}
  \| F^A \vp^A_k + F^B \vp^B_k + F^C \vp^C_k + \vp_\text{const} - \vp^D_k\|^2_2
\end{equation*}
for $K=K_0\cup K_1$.
Report the obtained values of $F^A$, $F^B$, $F^C$, and $\vp_\text{const}$.
Visualize the output of the fitted model~\cref{e_color_model} for all four
fragments together with the reference images~\tpD.
Some pixel intensities you obtain from~\cref{e_color_model} may lie outside
the range between 0 and 1, in which case you need to clip them,
e.g. using \verb`numpy.clip`. Calculate the error $S_{ABC}$ for each fragment,
i.e. report values of
$S_{ABC}(K_0)$, $S_{ABC}(K_1)$, $S_{ABC}(K_2)$, and $S_{ABC}(K_3)$.

\subsection{(b)\Pts{8}}
Repeat part~\Part{a} using only images \tpB as input. In this case, the model
simplifies to
\begin{equation}
 \vp_k = F^B \vp^B_k + \vp_\text{const},
\end{equation}
and the fitting error takes the form
\begin{equation*}
  S_{B}(K) = \frac{1}{|K|} \sum_{k\in K} \| F^B \vp^B_k + \vp_\text{const} -
  \vp^D_k\|^2_2.
\end{equation*}

\subsection{(c)\Pts{3}}
Compare results from parts~\Part{a} and \Part{b}. Does using multiple light
levels as input improve the quality of the fit compared to using a single light
level?

\section{P6. Determining hidden chemical sources\Pts{20}}
Suppose that $\rho(\vx,t)$
represents the concentration of a chemical diffusing in two-dimensional space,
where $\vx=(x,y)$. The concentration satisfies the diffusion equation
\begin{equation}
  \label{e_diff}
  \frac{\p \rho}{\p t} = \frac{\p^2 \rho}{\p x^2} + \frac{\p^2 \rho}{\p y^2}\,.
\end{equation}
If a localized point source of
chemical is introduced at the origin at $t=0$, its concentration satisfies
\begin{equation*}
  \rho_c(\vx,t) = \frac{1}{4\pi t} \exp \Big( -\frac{x^2+y^2}{4t}\Big).
\end{equation*}

\subsection{(a)\Pts{3}}
Show by direct calculation that the concentration
$\rho_c$ satisfies~\cref{e_diff}.

\subsection{(b)\Pts{7}}
Suppose that 49 point sources of chemicals are
introduced at $t=0$ with different strengths, on a $7 \times 7$ regular
lattice covering the coordinates $x=-3,-2,\ldots,3$ and
$y=-3,-2,\ldots,3$. By linearity of~\cref{e_diff} the concentration
will satisfy
\begin{equation*}
  \rho(\vx,t) = \sum_{k=0}^{48} \lambda_k \rho_c(\vx-\vec{v}_k,t),
\end{equation*}
where $\vec{v}_k$ is the $k$-th lattice site and $\lambda_k$ is the strength
of the chemical introduced at that site. In the archive \ZipName, you will find
a file \verb`p6_data.txt` with many measurements of $\rho(\vx,t)$ at~$t=4$.
The file contains three columns:
position~$x_i$, position~$y_i$, and concentration $\rho(\vx_i,4)$
for $i=0,\dots,199$.
By any means necessary, determine the concentration strengths $\lambda_k$.

\subsection{(c)\Pts{6}}
Suppose that the measurements have some experimental error, so that
the measured values $\tilde{\rho}_i$ in the file are related to the true
values $\rho_i$ according to
\begin{equation*}
  \tilde{\rho}_i = \rho_i + e_i
\end{equation*}
where the $e_i$ are normally distributed with mean $0$ and variance
$10^{-8}$. Construct a hypothetical sample of the true $\rho_i$, and
repeat your procedure from part~\Part{b} to determine the concentrations
$\lambda_k$. Repeat this sampling procedure for at least 100 times, and
use it to measure the standard deviation in the $\lambda_k$ at the
lattice sites $(0,0)$, $(1,1)$, $(2,2)$, $(3,3)$. Which of these has
the largest standard deviation and why?

\subsection{(d)\Pts{4}}
You should find that the concentrations $\lambda_k$ from part~\Part{b} take
values in the range between 0 and 32.
Round each $\lambda_k$ to the nearest integer and convert the integer
to its binary representation consisting of five bits.
The bits encode monochrome images.
To recover the first image, take the most significant (fifth bit, even if zero)
bit from the 5-bit binary
representation of all $\lambda_k$ and draw them on a $7x7$ grid.
Repeating this for all bits will give you five images.
What message can you read from them?

\end{document}
