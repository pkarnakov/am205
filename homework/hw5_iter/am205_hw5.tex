\documentclass[12pt]{article}
\usepackage{fullpage,amsmath,amsfonts,mathpazo,microtype,nicefrac}
\usepackage{xspace}

\usepackage[
  pdfusetitle=true, colorlinks=true, urlcolor=color2, linkcolor=color2,
  bookmarksdepth=1,
]{hyperref}

\usepackage{cleveref}
\crefname{equation}{}{}
\usepackage{titlesec}
% Section (problem) heading.
\titleformat{\section}{\large\bfseries}{}{0em}{}
\renewcommand{\thesection}{P\arabic{section}}
\crefname{section}{}{}
% Subsection (part) heading.
\titleformat{\subsection}[runin]{\normalfont\bfseries}{}{0em}{}
\renewcommand{\thesubsection}{(\alph{subsection})}
\crefname{subsection}{}{}

% Custom title page.
\makeatletter
\renewcommand\maketitle{
{\raggedright{\Large\bfseries\@title}\\[2ex]}}
\makeatother

% Macro definitions.
\newcommand{\N}{\mathbb{N}}
\newcommand{\Z}{\mathbb{Z}}
\newcommand{\Q}{\mathbb{Q}}
\newcommand{\R}{\mathbb{R}}
\newcommand{\p}{\partial}
\renewcommand{\vec}[1]{\mathbf{#1}}
\newcommand{\vx}{\vec{x}}
\newcommand{\vp}{\vec{p}}
\newcommand{\Trans}{\mathsf{T}}
\newcommand{\Optional}{Optional.}
\newcommand{\Part}[1]{\textbf{(#1)}}
\newcommand{\GitRoot}{https://github.com/pkarnakov/am205/tree/main}
\newcommand{\HrefPublic}[2]{\href{\GitRoot/\detokenize{#1}}{\normalfont\texttt{[\detokenize{#2}]}}}
\newcommand{\Pts}[1]{~[\textbf{#1 pts}]}

% Colors.
\usepackage{xcolor}
\definecolor{color0}{HTML}{FF1F5B}
\definecolor{color1}{HTML}{00CD6C}
\definecolor{color2}{HTML}{009ADE}
\definecolor{color3}{HTML}{AF58BA}
\definecolor{color4}{HTML}{FFC61E}
\definecolor{color5}{HTML}{F28522}
\definecolor{color6}{HTML}{A0B1BA}
\definecolor{color7}{HTML}{A6761D}
\definecolor{color8}{HTML}{E9002D}
\definecolor{color9}{HTML}{FFAA00}
\definecolor{color10}{HTML}{00B000}

\usepackage{graphicx}
\usepackage[section]{placeins}


\title{AM205 HW5. PDEs, optimization, iterative methods}

\newcommand{\PointsMax}{92\xspace}
\newcommand{\PointsCap}{40\xspace}

\begin{document}

\maketitle

\section{P1. Acoustics of Pierce Hall\Pts{40}}

Here you will solve the wave equation in a two-dimensional complex geometry.
The image below shows a plan of the third floor of Pierce Hall,
which will be your computational domain.
\begin{center}
  \includegraphics{p1_pierce.pdf}
\end{center}

A group of students hold an event in Pierce 301, the large room in the
bottom left of the plan. They set up a speaker shown in red.
The speaker creates acoustic waves propagating through the building,
possibly disturbing the occupants.
The wave equation
\begin{equation}
  p_{tt} - c^2 (p_{xx} + p_{yy})  = 0
\end{equation}
describes the propagation of acoustic waves
in the pressure field $p(x, y, t)$.
The pressure on the speaker equals $P(t)=P_0 \sin{\omega t}$.
The initial conditions are
$p(x, y, 0)=P(0)$ and $p_t(x, y, 0)=P'(0)$ inside the speaker,
and $p(x, y, 0)=0$ and $p_t(x, y, 0)=0$ everywhere else.
The boundary conditions are Dirichlet $p(x, y, t)=P(t)$ on the speaker
and zero Neumann $\mathbf{n} \cdot \nabla p=0$ on each wall, where $\mathbf{n}$
is a normal to the wall.

To discretize the problem, we will use a uniform grid of $N_x\times N_y$ cells
with $N_x=200$ and $N_y=100$.
Let $\Omega=\{(i,j) \;|\; 0\leq i < N_x, \; 0 \leq j < N_y\}$
denote the set of all cell indices.
Each cell $(i,j)$ is a square of size $h$ centered at $(x_i, y_j)$,
where $x_i=(i+0.5)h$ and $y_j=(j+0.5)h$.
The provided text file \verb`p1_pierce.txt` stores a matrix $M$
of size $N_x \times N_y$ representing the floor plan.
Each element corresponds to a cell of the grid.
Walls have values $M_{i,j}=0$ and open areas have $M_{i,j}=1$
for $(i,j)\in\Omega$.
The index of the bottom left corner cell is $(0, 0)$.
The index of the bottom right corner cell is $(N_x - 1, 0)$.
The speaker consists of cells
$S=\{(i,j) \;|\; 20\leq i < 24, \;20 \leq j < 24\}\subset \Omega$.

Use the following constants: $P_0=1$, $c=1$, $h=1$, $\Delta t = 0.5$,
and $\omega=0.67$.
These values correspond to acoustic waves at a frequency of about $100~\text{Hz}$ in air.

\subsection{(a)\Pts{5}} Read the matrix $M$ from \verb`p1_pierce.txt`.
Produce an image visualizing~$M$ and mark the location of the speaker on the
image.

\subsection{(b)\Pts{25}} Here we discretize the problem with
the following finite difference scheme
\begin{equation}
  \frac{p_{i,j}^{n+1} - 2p_{i,j}^n + p_{i,j}^{n-1}}{\Delta t^2}
  - c^2 \frac{p_{i+1,j}^n + p_{i,j+1}^n - 4p_{i,j}^n +
  p_{i-1,j}^n+p_{i,j-1}^n}{h^2} = 0,
\end{equation}
where $p^n_{i,j}\approx p(x_i, y_j, n\Delta t)$ are cell-centered values.
To discretize the boundary conditions on the walls,
we reformulate the scheme using two new variables
\begin{equation}
    u_{i,j}^n = 
    \begin{cases}
      (p^n_{i+1,j} - p^n_{i,j})/h, & M_{i+1,j} = M_{i,j}\\
      0,                           & M_{i+1,j} \neq M_{i,j}\\
    \end{cases}
\end{equation}
\begin{equation}
    v_{i,j}^n = 
    \begin{cases}
      (p^n_{i,j+1} - p^n_{i,j})/h, & M_{i,j+1} = M_{i,j}\\
      0,                           & M_{i,j+1} \neq M_{i,j}\\
    \end{cases}
\end{equation}
which approximate $p_x$ and $p_y$ taking into account the boundary conditions.
Now the update rule to obtain $p_{i,j}^{n+1}$ is
\begin{equation}
  \frac{p_{i,j}^{n+1} - 2p_{i,j}^n + p_{i,j}^{n-1}}{\Delta t^2}
  - c^2 \frac{u_{i,j}^n - u_{i-1,j}^n + v_{i,j}^n - v_{i,j-1}^n}{h} = 0
\end{equation}
for cells $(i,j)\notin S$.
Set $p_{i,j}^{n+1}=P((n+1) \Delta t)$ for cells $(i,j)\in S$,
i.e. inside the speaker.
Note that the scheme depends on values $p_{i,j}^n$ 
at $i=-1$, $i=N_x$, $j=-1$, and $j=N_y$.
You can either set them to zero or use periodic conditions.
The first two time levels $n=0$ and $n=1$ come from the initial
conditions
\begin{equation}
    p_{i,j}^n =
    \begin{cases}
      P(n\Delta t),  & (i,j)\in S\\
      0,             & (i,j)\notin S\\
    \end{cases}
\end{equation}
for cells $(i,j)\in\Omega$.
Write a program that implements the algorithm.

\subsection{(c)\Pts{5}}
Run your program from part \Part{b} until $n=5000$.
Visualize the pressure field $p^{n}_{i,j}$
at $n=1, 30, 100, 500, 1000$, and $5000$.
Qualitatively describe your results.

\subsection{(d)\Pts{5}}
The image above shows the locations of three measurement points $A$, $B$,
and~$C$.
They are located in cells with indices $(40, 49)$, $(168, 79)$, and~$(168, 10)$
respectively.
Run your program from part \Part{b} until $n=5000$.
For each of the three points, plot the signal $p^n_{i,j}$ as a function of $n$ 
and report the maximum achieved absolute value.
Qualitatively describe your results.

\section{P2. Bridge designer\Pts{30}}

This is a problem on PDE-constrained optimization.
Following the lecture
(\href{https://pkarnakov.github.io/am205/slides/unit4/#/134}{Unit~4, slide~135}),
consider a one-dimensional boundary value problem
\begin{equation}
\begin{aligned}
  -u'' + r(x,p) u &= f(x) \\
  u(0) &= 0 \\
  u(1) &= 0
\end{aligned}
  \label{e_primal}
\end{equation}
referred to as the \emph{primal problem}.
Here $u(x)$ is the unknown function in the domain $x\in[0,1]$,
$p\in\mathbb{R}^n$ is a parameter vector and
the other functions are
\begin{equation}
\begin{aligned}
  r(x, p) &=  \frac{A}{S} \sum_{i=1}^{n}e^{-\big(\frac{x - p_i}{S}\big)^2} \\
  f(x) &=  -1 \\
\end{aligned}
\end{equation}
with $n=2$, $A=20$, and $S=0.02$.
Let $u(x;p)$ denote the solution of the problem for a given~$p$.
One physical interpretation of this model is that~$u(x;p)$ is the height
of a bridge. The bridge is attached at the endpoints and
supported by~$n$ elastic pillars specified in~$r(x,p)$.
The bridge deforms under gravity specified in~$f(x)$.
The objective function is
\begin{equation}
  \mathcal{G}(p) = -\int_0^1 u(x;p) {\,\rm d}x,
  \label{e_obj}
\end{equation}
which is the negative mean value of the solution,
or the negative mean height of the bridge.

The task will be to minimize the objective function,
i.e. find the locations of the pillars that provide the largest mean height of
the bridge. You will need to solve various boundary value problems.
To discretize the equation, use a uniform grid with 1000 points
and second-order central differences for the second derivative.
To solve the resulting linear system, 
use a \textbf{sparse} linear solver,
e.g. TDMA or \verb`scipy.sparse.linalg.spsolve()`.
Use the trapezoid rule for the integrals such as~(\ref{e_obj}).

\subsection{(a)\Pts{6}}
Solve the primal problem for $p=p_\mathrm{init}=(0.2, 0.5)$.
Plot the solution $u(x;p)$ and the coefficient $r(x,p)$ as functions of $x$.
The expected result is shown below.

\begin{center}
\includegraphics{p2_init.pdf}
\end{center}

\subsection{(b)\Pts{4}}
Compute the derivatives of the objective function at $p=p_\mathrm{init}$
using forward differences
\begin{equation}
  \frac{\partial \mathcal{G}}{\partial p_i} \approx 
  \frac{\mathcal{G}(p + \delta e_i) - \mathcal{G}(p)}{\delta}
  \qquad i=1,\dots, n
\end{equation}
with $\delta=10^{-5}$.
Report the obtained values of
$\frac{\partial \mathcal{G}}{\partial p_i}$ up to 16 significant digits.

\subsection{(c)\Pts{7}}
Compute the derivatives of the objective function at $p=p_\mathrm{init}$
using the \emph{direct method} discussed in the lecture.
In particular, differentiate the primal equation~(\ref{e_primal})
with respect to~$p_i$ to derive a boundary value problem 
for $\frac{\partial u}{\partial p_i}$,
solve the problem and evaluate the integral
\begin{equation}
  \frac{\partial \mathcal{G}(p)}{\partial p_i}
  =
  \int_0^1 \frac{\partial u}{\partial p_i} {\,\rm d}x
  \qquad i=1,\dots, n.
\end{equation}
Report the obtained values of
$\frac{\partial \mathcal{G}}{\partial p_i}$ up to 16 significant digits.

\subsection{(d)\Pts{8}}
Compute the derivatives of the objective function at $p=p_\mathrm{init}$
using the \emph{adjoint method} discussed in the lecture.
In particular, solve the adjoint problem to obtain $z(x;p)$
such that the derivatives are given by
\begin{equation}
  \frac{\partial \mathcal{G}(p)}{\partial p_i}
  =
  - \int_0^1 \frac{\partial r(x,p)}{\partial p_i} z(x;p) u(x;p) \text{d}x.
\end{equation}
Report the obtained values of
$\frac{\partial \mathcal{G}}{\partial p_i}$ up to 16 significant digits.

\subsection{(e)\Pts{5}}
Use a gradient-based optimizer, e.g. BFGS in \verb`scipy.optimize.minimize()`,
to minimize $\mathcal{G}(p)$.
To compute the gradients, choose any of the methods
you have implemented in parts \Part{b}, \Part{c}, or \Part{d}.
Report the optimal parameters $p_\mathrm{opt}$.
Plot the solution $u(x;p)$ and the coefficient $r(x,p)$
at $p=p_\mathrm{opt}$ as functions of $x$.
The expected outcome is that the pillars are distributed in the
domain more evenly and the mean height of the bridge increases.


\section{P3. Conjugate gradients\Pts{22}}

Consider a boundary value problem for the discrete Laplace equation
\begin{equation}
  \begin{aligned}
    -u_{i-1} + 2u_i - u_{i+1} &= 0 \qquad i=1,\dots,n-1 \\
    u_0 &= a \\
    u_n &= b \\
  \end{aligned}
  \label{e_laplace}
\end{equation}
with $a=-1$ and $b=2$.
Assume that the grid points are $x_i=i/n$ for $i=0,\dots,n$.
This linear problem can be written in matrix form as $Au=f$ with
the matrix $A\in\mathbb{R}^{(n+1)\times(n+1)}$ and
vectors~$u,f\in\mathbb{R}^{(n+1)}$ defined as
\begin{equation}
  A =
  \left[
  \begin{array}{cccccc}
  1  & 0  &         &        &      &         \\
  0  & 2  & -1      &        &      &         \\
     & -1 & \ddots  & \ddots &      &         \\
     &    & \ddots  & \ddots & -1   &         \\
     &    &         & -1     & 2    & 0        \\
     &    &         &        & 0    & 1       \\
  \end{array}
  \right],
\quad 
  u =
  \left[
  \begin{array}{c}
    u_0\\
    u_1\\
    u_2\\
    \vdots \\
    u_{n-2}\\
    u_{n-1}\\
    u_n\\
  \end{array}
  \right],
\quad 
  f =
  \left[
  \begin{array}{c}
    a\\
    a\\
    0\\
    \vdots \\
    0\\
    b\\
    b\\
  \end{array}
  \right].
\end{equation}
Note that $A$ is symmetric and sparse.
However, \textbf{avoid constructing the matrix} $A$ in the following tasks,
you should only rely on matrix-vector products.

\subsection{(a)\Pts{2}}
Analytically verify that $Au=f$ is indeed equivalent to~(\ref{e_laplace}).

\subsection{(b)\Pts{3}}
Write a function \verb`mul_A(u)` that takes a vector $u$ and returns the product $Au$.
The algorithmic complexity of the function needs to be $\mathcal{O}(n)$.

\subsection{(c)\Pts{8}}
Implement the conjugate gradient method to solve a linear system $Au=f$.
Your implementation should use \verb`mul_A(u)` without accessing the matrix~$A$
directly.

\subsection{(d)\Pts{5}}
Set $n=32$ and solve the equation $Au=f$ using your implementation from
part~\Part{c}. 
Let $u^{(k)}$ denote the solution at iteration $k$ of the method.
For the initial guess use
\begin{equation}
    u^{(0)}_i=
    \begin{cases}
      a, & i=0 \\
      b, & i=n \\
      0, & \text{otherwise}
    \end{cases}
\end{equation}
which satisfies the boundary conditions.
Perform $n + 1$ iterations.
Make a log-linear plot of the norm of the residual $\|Au^{(k)}-f\|_2$ as a function of $k$.
Qualitatively describe your results.
How many iterations does it take to achieve $\|Au^{(k)}-f\|_2<10^{-10}$?

\subsection{(e)\Pts{4}}
Plot $u^{(k)}_i$ from part~\Part{d} as a function of $x_i$ for $k=0, 1, \dots, n$.
Showing only those points for which $|(Au^{(k)}-f)_i|>10^{-2}$,
make a log-linear plot of $|(Au^{(k)}-f)_i|$ as a function of $x_i$ for $k=0, 1,
\dots, n$.
Qualitatively describe your results.
Do you observe two different trends between the cases of $k<n/2$ and $k>n/2$?


\end{document}
